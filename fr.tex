%%%%%%%%%%%%%%%%%
% This is an sample CV template created using altacv.cls
% (v1.2, 11 February 2020) written by LianTze Lim (liantze@gmail.com). Now compiles with pdfLaTeX, XeLaTeX and LuaLaTeX.
%
%% It may be distributed and/or modified under the
%% conditions of the LaTeX Project Public License, either version 1.3
%% of this license or (at your option) any later version.
%% The latest version of this license is in
%%    http://www.latex-project.org/lppl.txt
%% and version 1.3 or later is part of all distributions of LaTeX
%% version 2003/12/01 or later.
%%%%%%%%%%%%%%%%

%% If you need to pass whatever options to xcolor
\PassOptionsToPackage{dvipsnames}{xcolor}

%% If you are using \orcid or academicons
%% icons, make sure you have the academicons
%% option here, and compile with XeLaTeX
%% or LuaLaTeX.
% \documentclass[10pt,a4paper,academicons]{altacv}

%% Use the "normalphoto" option if you want a normal photo instead of cropped to a circle
% \documentclass[10pt,a4paper,normalphoto]{altacv}

\documentclass[10pt,a4paper,ragged2e]{altacv}

%% AltaCV uses the fontawesome and academicon fonts
%% and packages.
%% See http://texdoc.net/pkg/fontawesome and http://texdoc.net/pkg/academicons for full list of symbols. You MUST compile with XeLaTeX or LuaLaTeX if you want to use academicons.

% Change the page layout if you need to
\geometry{left=1.25cm,right=1.25cm,top=1.5cm,bottom=1.5cm,columnsep=1.2cm}

% The paracol package lets you typeset columns of text in parallel
\usepackage{paracol}

% Change the font if you want to, depending on whether
% you're using pdflatex or xelatex/lualatex
\ifxetexorluatex
  % If using xelatex or lualatex:
  \setmainfont{Lato}
\else
  % If using pdflatex:
  \usepackage[utf8]{inputenc}
  \usepackage[T1]{fontenc}
  \usepackage[default]{lato}
\fi

% Change the colours if you want to
\definecolor{Orange}{HTML}{FF8C00}
\definecolor{myRed}{HTML}{ff0000}
\definecolor{Mulberry}{HTML}{72243D}
\definecolor{SlateGrey}{HTML}{2E2E2E}
\definecolor{LightGrey}{HTML}{666666}
\definecolor{myBlack}{HTML}{000000}
\colorlet{heading}{myBlack}
\colorlet{accent}{Orange}
\colorlet{emphasis}{myBlack}
\colorlet{body}{LightGrey}

% Change the bullets for itemize and rating marker
% for \cvskill if you want to
\renewcommand{\itemmarker}{{\small\textbullet}}
\renewcommand{\ratingmarker}{\faCircle}

%% sample.bib contains your publications
\addbibresource{sample.bib}
\usetikzlibrary{shapes}


\begin{document}

%\makecvfooter
%  {\today}
%  {Resume}
%  {\thepage}
  
\name{Fakher HANNAFI}
\tagline{\textcolor{LightGrey}{
 Consultant Cloud/Data Chez Devoteam Revolve}}

%% You can add multiple photos on the left or right
\photoR{3cm}{photo_aws}
% \photoL{2.5cm}{Yacht_High,Suitcase_High}
\personalinfo{
\printinfo{\faBirthdayCake}{26 ans} \hspace{24mm} \location{Grand Paris, France} 
 \hspace{9,5mm} \phone{+33-7-8491-4455}
  %
  % Not all of these are required!
  % You can add your own with \printinfo{symbol}{detail}
 
  \email{fakher.hannafi@gmail.com}
  \linkedin{linkedin.com/in/fakher-hannafi}
  \github{github.com/fakhrovski}

 
  %\mailaddress{}
  %\homepage{www.homepage.com/}
  %\twitter{@fakhrovski}
  %% You MUST add the academicons option to \documentclass, then compile with LuaLaTeX or XeLaTeX, if you want to use \orcid or other academicons commands.
  % \orcid{orcid.org/0000-0000-0000-0000}
}

\makecvheader

%% Depending on your tastes, you may want to make fonts of itemize environments slightly smaller
% \AtBeginEnvironment{itemize}{\small}

%% Set the left/right column width ratio to 6:4.
\columnratio{0.6}

% Start a 2-column paracol. Both the left and right columns will automatically
% break across pages if things get too long.
\begin{paracol}{2}
\cvsection{Expérience}

\cvevent{Consultant Cloud/DATA }{\textcolor{myBlack}{GRTGAZ} \hspace{7,7cm}\includegraphics[width=0.05 \textwidth]{440px-Logo_GRT_Gaz.svg.png}}{Septembre 2020 -- Décembre 2020}{}
\divider

\cvevent{Tech Lead }{\textcolor{myBlack}{Groupe BPCE} \hspace{6,8cm}\includegraphics[width=0.07 \textwidth]{bpce}}{Avril 2019 -- Octobre 2020}{}

\divider

\cvevent{Développeur Web}{\textcolor{LightGrey}{ORANGE} \hspace{7.8cm}\includegraphics[width=0.03 \textwidth]{orange}}{Mai 2018 -- Novembre 2018}{}
%\begin{itemize}
%\item Développeur Web
%\item Ingénieur Devops
%\end{itemize}

\cvsection{Projets}

\cvevent{Remise en Conformité des Environnements Data Science sur AWS}{Consultant DataOps}{Octobre 2020 -- Décembre 2020}{}
\begin{itemize}
\item Mettre en place un cadre de gouvernance sur les environnements Data Science tout en respectant les exigences sécurité de GRTGAZ et les recommandations/bonnes pratiques de AWS
\item \textbf{Mots clés:} IAM, Détection, Protection de l'infrastructure, Protection des données, Gestion des incidents
\item \textbf{Technologies:} Cloud Custodian, Inspector, Config, SSM, Endpoints, VPC
\end{itemize}

\divider

\cvevent{Industrialiser les modèles de Data Science dans le Groupe}{Chef de Projet/Architecte DATA}{Feb 2020 -- Octobre 2020}{}
\begin{itemize}
\item Déployer les modèles de Data Science dans les environnements de production.
\item \textbf{Mots clés:} Excellence opérationnelle, Monitoring, API/Batch Patterns, CI/CD, Containers
\item \textbf{Technologies:} Anaconda, Hortonworks Data Platform, Artifactory, Elastic, Jenkins, XLR/XLD
\end{itemize}

\divider
\cvevent{Migration de la plateforme Big Data Groupe}{Ingénieur Infrastructure}{Janv 2020 -- 
Octobre 2020}{}
\begin{itemize}
\item \textbf{Mots Clés:} conteneurisation/virtualisation, ESX, SAS, MySQL, PRA/PCA, Availability, Fault Tolerance, SLA, Power BI, HDP 2.5.6
\medskip
\end{itemize}

\divider
\cvevent{Outillage pour l'automatisation}{Développeur Web | Développeur Data}{Avril 2019 -- Octobre 2020}{}
\begin{itemize}
\item Développer et déployer 3 outils internes de bout en bout.
\item \textbf{Technologies:} Angular 8, Nodejs 10, Oracle 12c, PowerShell, Power BI, Jenkins
\end{itemize}
\medskip


\newpage

\cvevent{Migration des données vers le cloud AWS}{AWS Cloud Engineer}{Octobre 2019 -- Décembre 2019}{}
\item Homologation de la chaine de traitement de données. Vérification de la robustesse et la performance de la solution.
\item \textbf{Technologies:} S3, Lambda, Athena, Cloudwatch, IAM, STS, SNS
\medskip

\cvevent{Outil de Monitoring}{Développeur Full Stack | Ingénieur Devops}{Avril 2019 -- En cours}{}
\item Concevoir, développer, tester, déployer et documenter un outil de supervision qui permet de sortir des KPIs de la messagerie professionnelle Orange
\item \textbf{Mots Clés:} NodeJS, VueJs, JEST, Bootstrap, Rest API, Websocket, Git, Openshift, Jenkins, Kanban
\medskip

\cvsection{Centres d'intérêt}

% Adapted from @Jake's answer from http://tex.stackexchange.com/a/82729/226
% \wheelchart{outer radius}{inner radius}{
% comma-separated list of value/text width/color/detail}

\wheelchart{1.5cm}{0.5cm}{%
  6/8em/accent!30/{Séries télévisées},
  3/8em/accent!40/Tennis de table,
  8/8em/accent!60/Musique,
  2/10em/accent/Coding,
  5/6em/accent!20/Networking
}

% use ONLY \newpage if you want to force a page break for
% ONLY the current column


%% Switch to the right column. This will now automatically move to the second
%% page if the content is too long.
\switchcolumn

\cvsection{Biographie}


\begin{quote}
\textcolor{LightGrey}{``The more I live, the more I learn. The more I learn, the more I realize, the less I know''}
\end{quote}


\cvsection{Certifications}

\cvachievement{\faTrophy}{Certifications techniques}{}


\includegraphics[width=0.1 \textwidth]{clf} \includegraphics[width=0.1 \textwidth]{dva} \includegraphics[width=0.1 \textwidth]{saa} \includegraphics[width=0.1 \textwidth]{soa}\\

\includegraphics[width=0.1 \textwidth]{azure-900.png}
\includegraphics[width=0.1\textwidth]{azure-administrator-associate.png}
\includegraphics[width=0.1\textwidth]{azure-developer-associate-600x600.png}
\includegraphics[width=0.1\textwidth]{CERT-Expert-DevOps-Engineer-600x600.png}
\includegraphics[width=0.1\textwidth]{Terraform-Associate-Badge.png}
\includegraphics[width=0.1\textwidth]{ckad_from_cncfsite.png}
\includegraphics[width=0.1\textwidth]{aceAssociatetBadgeArtboard_1.png}
\includegraphics[width=0.1\textwidth]{CERT-Associate-Data-Analyst-600x600.png}
\includegraphics[width=0.1\textwidth]{jenkins}

\divider


\cvachievement{\faTrophy}{Certifications non techniques}{}
\includegraphics[width=0.16 \textwidth]{itil} \includegraphics[width=0.06 \textwidth]{psm}
\hspace{5mm}\includegraphics[width=0.12 \textwidth]{toeic}

\divider\\
\cvachievement{\faTrophy}{Diplôme national de Musique}{Professeur, Violoniste, Pianiste}

\cvsection{Compétences}

\cvtag{Autonome} \cvtag{Challenger}\cvtag{Proactif}\\
\cvtag{Apprenant rapide} 


\divider\smallskip\\
\cvtag{Data}\cvtag{Devops}\cvtag{Cloud}\cvtag{Web}\\

\divider\smallskip\\

\cvtag{CI/CD}\cvtag{IaC}\cvtag{VCS}\cvtag{Planning}\cvtag{Testing}\cvtag{Monitoring}\cvtag{Containers}\cvtag{Serverless}\\

\divider\smallskip\\

\cvtag{Angular}\cvtag{Node.js}\cvtag{Python}\cvtag{Powershell}\cvtag{Vtom}\cvtag{Redhat}\cvtag{Jenkins}\cvtag{Ansible}\cvtag{Terraform}\cvtag{Docker}\cvtag{Elasticsearch}\cvtag{Power BI}\cvtag{Grafana}\cvtag{Hadoop}\cvtag{Spark}\cvtag{Oracle}\\
\divider\smallskip\\
\cvtag{Scrum}\cvtag{ITIL4}\\

\break
\cvsection{Langages}

\cvskill{Français}{5}
\divider

\cvskill{Arabe}{5}
\divider

\cvskill{Anglais}{4}

%% Yeah I didn't spend too much time making all the
%% spacing consistent... sorry. Use \smallskip, \medskip,
%% \bigskip, \vpsace etc to make ajustments.
\medskip

\cvsection{Cursus Scolaire}

\cvevent{IT Engineering}{Ecole supérieur de communication de Tunis, Sup'COM \break
\location{\textcolor{LightGrey}{Tunisia}}}{Sept 2015 -- Juin 2018}{}
\item Spécialité: Service Télécom Cloud

\divider

\cvevent{Etudes d'ingénieur }{Institut Préparatoire des Etudes d'Ingénieurie à Tunis, IPEIT \break
\location{\textcolor{LightGrey}{Tunisie}}}{Sept 2013 -- Juin 2015}{}
\item Spécialité: Mathématiques & Physiques \\ Rang: 135/2200 dans le concours national

\divider

\cvevent{Baccalauréat Scientifique}{Lycée Pilote Kairouan \break
\location{\textcolor{LightGrey}{Tunisie}}}{Sept 2009 -- Juin 2013}{}
\item Spécialité: Mathématiques, Moyenne: 16.56/20 \\
Mention: Très Bien

\divider

\cvevent{Diploma of Oriental Music}{National Conservatory \break
\location{\textcolor{LightGrey}{Tunisia}}}{Sept 2005 -- Juin 2010}{}
\item Spécialité: Folklore tunisien, oriental \\
Mention: Bien



\end{paracol}



\end{document}
