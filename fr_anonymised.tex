%%%%%%%%%%%%%%%%%
% This is an sample CV template created using altacv.cls
% (v1.2, 11 February 2020) written by LianTze Lim (liantze@gmail.com). Now compiles with pdfLaTeX, XeLaTeX and LuaLaTeX.
%
%% It may be distributed and/or modified under the
%% conditions of the LaTeX Project Public License, either version 1.3
%% of this license or (at your option) any later version.
%% The latest version of this license is in
%%    http://www.latex-project.org/lppl.txt
%% and version 1.3 or later is part of all distributions of LaTeX
%% version 2003/12/01 or later.
%%%%%%%%%%%%%%%%

%% If you need to pass whatever options to xcolor
\PassOptionsToPackage{dvipsnames}{xcolor}

%% If you are using \orcid or academicons
%% icons, make sure you have the academicons
%% option here, and compile with XeLaTeX
%% or LuaLaTeX.
% \documentclass[10pt,a4paper,academicons]{altacv}

%% Use the "normalphoto" option if you want a normal photo instead of cropped to a circle
\documentclass[10pt,a4paper,normalphoto]{altacv}

\documentclass[10pt,a4paper,ragged2e]{altacv}

%% AltaCV uses the fontawesome and academicon fonts
%% and packages.
%% See http://texdoc.net/pkg/fontawesome and http://texdoc.net/pkg/academicons for full list of symbols. You MUST compile with XeLaTeX or LuaLaTeX if you want to use academicons.

% Change the page layout if you need to
\geometry{left=1.25cm,right=1.25cm,top=1.5cm,bottom=1.5cm,columnsep=1.2cm}

% The paracol package lets you typeset columns of text in parallel
\usepackage{paracol}

% Change the font if you want to, depending on whether
% you're using pdflatex or xelatex/lualatex
\ifxetexorluatex
  % If using xelatex or lualatex:
  \setmainfont{Lato}
\else
  % If using pdflatex:
  \usepackage[utf8]{inputenc}
  \usepackage[T1]{fontenc}
  \usepackage[default]{lato}
\fi

% Change the colours if you want to
\definecolor{Orange}{HTML}{FF8C00}
\definecolor{myRed}{HTML}{ff0000}
\definecolor{Mulberry}{HTML}{72243D}
\definecolor{SlateGrey}{HTML}{2E2E2E}
\definecolor{LightGrey}{HTML}{666666}
\definecolor{myBlack}{HTML}{000000}
\colorlet{heading}{myBlack}
\colorlet{accent}{Orange}
\colorlet{emphasis}{myBlack}
\colorlet{body}{LightGrey}

% Change the bullets for itemize and rating marker
% for \cvskill if you want to
\renewcommand{\itemmarker}{{\small\textbullet}}
\renewcommand{\ratingmarker}{\faCircle}

%% sample.bib contains your publications
\addbibresource{sample.bib}
\usetikzlibrary{shapes}


\begin{document}

%\makecvfooter
%  {\today}
%  {Resume}
%  {\thepage}
  
\name{Fakher HANNAFI}
\tagline{\textcolor{LightGrey}{
Architecte Data/ML || 4 ans d'Expérience }}

%% You can add multiple photos on the left or right
\photoR{3cm}{photo_aws}
%\photoL{2.5cm}{Yacht_High,Suitcase_High}
\personalinfo{
\printinfo{\faBirthdayCake}{27 ans} \hspace{24mm} \location{Grand Paris, France} 
 %\hspace{9,5mm} \phone{+33-7-8491-4455}
  %
  % Not all of these are required!
  % You can add your own with \printinfo{symbol}{detail}
 
  \email{fakher.hannafi@gmail.com}
  \linkedin{linkedin.com/in/fakher-hannafi}
  \github{github.com/fakherhannafi}
  \homepage{www.fakher-hannafi.com}
  \homepage{fakherhannafi.medium.com/}

 
  %\mailaddress{}
  %\homepage{www.homepage.com/}
  %\twitter{@fakhrovski}
  %% You MUST add the academicons option to \documentclass, then compile with LuaLaTeX or XeLaTeX, if you want to use \orcid or other academicons commands.
  % \orcid{orcid.org/0000-0000-0000-0000}
}

\makecvheader

%% Depending on your tastes, you may want to make fonts of itemize environments slightly smaller
% \AtBeginEnvironment{itemize}{\small}

%% Set the left/right column width ratio to 6:4.
\columnratio{0.6}

% Start a 2-column paracol. Both the left and right columns will automatically
% break across pages if things get too long.
\begin{paracol}{2}
\cvsection{Expérience}

\cvevent{Architecte Data}{\textcolor{myBlack}{BPI France} \hspace{6.6cm}\includegraphics[width=0.08 \textwidth]{bpifrance.png}}{Décembre 2021 -- Aujourd'hui}{}
\divider

\cvevent{Architecte Data}{\textcolor{myBlack}{ENGIE} \hspace{7.5cm}\includegraphics[width=0.07 \textwidth]{engie.png}}{Octobre 2021 -- Décembre 2021}{}
\divider

\cvevent{AWS Data/ML Engineer}{\textcolor{myBlack}{GRTGAZ} \hspace{7cm}\includegraphics[width=0.07 \textwidth]{440px-Logo_GRT_Gaz.svg.png}}{Octobre 2020 -- Octobre 2021}{}
\divider

\cvevent{Community Builder, Speaker, Blogger}{\textcolor{myBlack}{Devoteam Revolve} \hspace{3.6cm}\includegraphics[width=0.18 \textwidth]{devoteam_revolve.png}}{Octobre 2020 -- Décembre 2021}{}
\divider

\cvevent{Tech Lead Data}{\textcolor{myBlack}{Groupe BPCE} \hspace{6,2cm}\includegraphics[width=0.09 \textwidth]{bpce}}{Avril 2019 -- Septembre 2020}{}

\divider

\cvevent{Développeur Full stack}{\textcolor{LightGrey}{ORANGE FR} \hspace{7.1cm}\includegraphics[width=0.037 \textwidth]{orange}}{Mai 2018 -- Mars 2019}{}
%\begin{itemize}
%\item Développeur Web
%\item Ingénieur Devops
%\end{itemize}

\cvsection{Projets}

\cvevent{Design, Build et Run des Pipelines Data}{Architecte Data - BPI France}{Décembre 2021 -- Aujourd'hui}{}
\begin{itemize}
    \item Concevoir, Développer et Monitorer des pipelines Data pour le comptoir des données Client.
    \item \textbf{Mots clés: } Data Ingestion, Data Orchestration, Data Quality, Data Monitoring
    \item \textbf{Technologies:} Gitlab, Glue, Spark, Python, Airflow, Cloudwatch, Datadog, Jenkins.
\end{itemize}
\divider

\cvevent{Architecture Cloud, Governance Data, Data Pipelines}{Architecte Data AWS - ENGIE}{Octobre 2021 -- Décembre 2021}{}
\begin{itemize}
    \item Design et Implémentation d'un MVP d'une solution de Data Gouvernance basée sur le le software Collibra.
    \item Support Adoption de la plateforme CDH, PowerBI pour toutes les entités/BUs du groupe.
    \item \textbf{Mots clés: } Data Gouvernance, Data Quality, Ontology
    \item \textbf{Technologies:} Git, Cloudformation, EC2, Palantir Foundry, CDH
\end{itemize}
\divider

\cvevent{Prévisions des consommations de Gaz PRECO}{Architecte Data/ML - GRTGAZ}{Janvier 2021 -- Aujourd'hui}{}
\begin{itemize}
\item Design et implémentation d'une architecture d'une solution ML pour assurer le le déploiement et le monitoring des modèles PRECO
\item \textbf{Mots clés: } Data Validation, Feature Engineering, Experimentation Tracking, Data Versioning, Model Monitoring, Feedback loop
\item \textbf{Technologies:} Git, CodePipeline, Sagemaker, Cloudwatch, IAM, SNS, AWS Workspaces
\end{itemize}
\divider

\cvevent{Remise en Conformité des Environnements Data Science sur AWS}{Consultant Cloud/Data - GRTGAZ}{Octobre 2020 -- Décembre 2020}{}
\begin{itemize}
\item Mettre en place un cadre de gouvernance sur les environnements Data Science tout en respectant les exigences sécurité de GRTGAZ et les recommandations/bonnes pratiques de AWS
\item \textbf{Mots clés:} IAM, Détection, Protection de l'infrastructure, Protection des données, Gestion des incidents
\item \textbf{Technologies:} Cloud Custodian, AWS Inspector, Config, SSM,EC2, VPC Endpoints, Cloudwatch
\end{itemize}
\divider

\cvevent{Community Builder, Speaker}{Consultant Data AWS - Devoteam Revolve}{Janvier 2021 -- Aujourd'hui}{}
\begin{itemize}
    \item Help junior consultants onboard on AWS and Data, Speaker in Salon Big Data in 2021 Edition,  Write Blogs in Enterprise Blog Site, Build Data Offer Proposition for the company.
\end{itemize}
\divider

\cvevent{Migration des données vers le cloud AWS}{AWS Cloud Engineer- BPCE}{Octobre 2019 -- Décembre 2019}{}
\item Homologation de la chaine de traitement de données. Vérification de la robustesse et la performance de la solution.
\item \textbf{Technologies:} S3, Lambda, Athena, Cloudwatch, IAM, STS, SNS
\medskip

\divider
\cvevent{Pilote d'infrastructure d'une plateforme de sécurisation d'accès à la donnée avec DENODO}{Architecte Data - BPCE}{Mai 2020 -- Octobre 2020}{}
\begin{itemize}
\item Déploiement d'une solution de Data virtualisation DENODO qui permet essentiellement de sécuriser les accès à la donnée au sein du groupe bancaire BPCE
\item \textbf{Mots clés:} Haute Disponibilité, Centralisation des accès, Business Continuity Plan, Gouvernance de la donnée  
\item \textbf{Technologies:} Denodo 8, MyCloud, Redhat 7, Elastisearch
\medskip
\end{itemize}
\divider

\cvevent{Industrialiser les modèles de Data Science dans le Groupe}{Chef de Projet - Architecte DATA- BPCE}{Feb 2020 -- Octobre 2020}{}
\begin{itemize}
\item Déployer les modèles de Data Science dans les environnements de production Big Data.
\item \textbf{Mots clés:} Excellence opérationnelle, Monitoring, API/Batch Patterns, CI/CD, Containers
\item \textbf{Technologies:} Anaconda, Hortonworks Data Platform, Artifactory, Elastic, Jenkins, XLR/XLD
\medskip
\end{itemize}

\divider
\cvevent{Migration de la plateforme Big Data Groupe}{Ingénieur Infrastructure - BPCE}{Janv 2020 -- 
Octobre 2020}{}
\begin{itemize}
\item Migration de la plateforme Big DATA Groupe de HDP 2.5 vers HDP 3.1
\item \textbf{Mots Clés:} conteneurisation/virtualisation, ESX, SAS, MySQL, PRA/PCA, Availability, Fault Tolerance, SLA, Power BI, HDP 2.5.6
\end{itemize}

\divider
\cvevent{Outillage pour l'automatisation}{Développeur Web | Développeur Data - BPCE}{Avril 2019 -- Octobre 2020}{}
\begin{itemize}
\item Développer et déployer 3 outils internes de bout en bout.
\item \textbf{Technologies:} Angular 8, Nodejs 10, Oracle 12c, PowerShell, Power BI, Jenkins
\end{itemize}
\medskip

\cvevent{Outil de Monitoring}{Développeur Full Stack | Ingénieur Devops - Orange}{Mai 2018 -- Mars 2019}{}
\item Concevoir, développer, tester, déployer et documenter un outil de supervision qui permet de sortir des KPIs de la messagerie professionnelle Orange
\item \textbf{Mots Clés:} NodeJS, VueJs, JEST, Bootstrap, Rest API, Websocket, Git, Openshift, Jenkins, Kanban
\medskip

\cvsection{Centres d'intérêt}

% Adapted from @Jake's answer from http://tex.stackexchange.com/a/82729/226
% \wheelchart{outer radius}{inner radius}{
% comma-separated list of value/text width/color/detail}

\wheelchart{1.5cm}{0.5cm}{%
  6/8em/accent!30/{Séries télévisées},
  3/8em/accent!40/Tennis de table,
  8/8em/accent!60/Musique,
  2/10em/accent/Coding,
  5/6em/accent!20/Networking
}

% use ONLY \newpage if you want to force a page break for
% ONLY the current column


%% Switch to the right column. This will now automatically move to the second
%% page if the content is too long.
\switchcolumn

\cvsection{Biographie}


\begin{quote}
\textcolor{LightGrey}{``Passionné par la Data et le DevOps dans le Cloud. AWS, Azure, Hashicorp... avide de nouvelles connaissances, 22 certifications professionnelles en 2 ans! Intéressé par les sujets d'industrialisation Data Science et au ML Ops, je suis aussi prof de musique, violoniste et amateur du tennis de table.''}
\end{quote}

\cvsection{Compétences}

\cvtag{Autonome} \cvtag{Challenger}\cvtag{Proactif}\\
\cvtag{Apprenant rapide} 

\divider\smallskip\\
\cvtag{High Availability}\cvtag{Fault Tolerance}\cvtag{Scalability}\cvtag{Operational Excellence}\cvtag{Performance}\cvtag{Security}\\
\cvtag{SQL} \cvtag{NoSQL}\cvtag{Spark}\cvtag{Python}\cvtag{Data warehousing}\cvtag{Data Lake} \cvtag{Data Streaming} \cvtag{Cloud Data Platforms} \cvtag{Data Virtualization} \cvtag{Data Security}\cvtag{Data Integration}\cvtag{Data Ingestion} \cvtag{ML Operations} \\\cvtag{Elastic}\cvtag{MongoDB}\cvtag{Grafana}\cvtag{Denodo}\cvtag{Power BI}\cvtag{Splunk}\cvtag{Datadog}\cvtag{Hadoop}\cvtag{Kafka}\cvtag{Oracle}\cvtag{Powershell}\cvtag{Redhat}\cvtag{Jenkins}\cvtag{Ansible}\cvtag{Terraform}

\divider\smallskip\\

\cvtag{CI/CD}\cvtag{Terraform}\cvtag{Vault}\cvtag{Git}\cvtag{Kubernetes}\cvtag{Docker}

\divider\smallskip\\
\cvtag{Scrum}\cvtag{ITIL4}\cvtag{SAFE}

\cvsection{Certifications}

\cvachievement{\faTrophy}{Certifications techniques}{}

 \includegraphics[width=0.105 \textwidth]{devops} \includegraphics[width=0.105 \textwidth]{data-analytics.png}
\includegraphics[width=0.1 \textwidth]{clf} \includegraphics[width=0.1 \textwidth]{dva} \includegraphics[width=0.1 \textwidth]{saa} \includegraphics[width=0.1 \textwidth]{soa}\\

\includegraphics[width=0.105\textwidth]{azure-solutions-architect-expert-600x600.png}
\includegraphics[width=0.1\textwidth]{CERT-Expert-DevOps-Engineer-600x600.png}
\includegraphics[width=0.105\textwidth]{data-eng-az}
\includegraphics[width=0.1\textwidth]{CERT-Associate-Data-Analyst-600x600.png}
\includegraphics[width=0.1\textwidth]{azure-administrator-associate.png}
\includegraphics[width=0.1\textwidth]{azure-developer-associate-600x600.png}
\includegraphics[width=0.105\textwidth]{security-engineer.png}
\includegraphics[width=0.1 \textwidth]{azure-900.png}
\includegraphics[width=0.105\textwidth]{data-fund}
\includegraphics[width=0.1\textwidth]{ckad_from_cncfsite.png}
\includegraphics[width=0.104\textwidth]{cka.png}

\includegraphics[width=0.1\textwidth]{Terraform-Associate-Badge.png}
\includegraphics[width=0.105\textwidth]{vault}
\includegraphics[width=0.1\textwidth]{aceAssociatetBadgeArtboard_1.png}
\includegraphics[width=0.09\textwidth]{jenkins}
\hspace{0,6mm}



\divider


\cvachievement{\faTrophy}{Certifications non techniques}{}
\includegraphics[width=0.16 \textwidth]{itil} \includegraphics[width=0.06 \textwidth]{psm}
\hspace{5mm}\includegraphics[width=0.12 \textwidth]{toeic}

\divider\\
\cvachievement{\faTrophy}{Diplôme national de Musique}{Professeur, Violoniste, Pianiste}



\cvsection{Langues}

\cvskill{Français}{5}
\divider

\cvskill{Arabe}{5}
\divider

\cvskill{Anglais}{4}

%% Yeah I didn't spend too much time making all the
%% spacing consistent... sorry. Use \smallskip, \medskip,
%% \bigskip, \vpsace etc to make ajustments.
\medskip

\cvsection{Cursus Scolaire}

\cvevent{Etudes d'ingénieur en TIC}{Ecole supérieur de communication de Tunis, Sup'COM \break
\location{\textcolor{LightGrey}{Tunisie}}}{Sept 2015 -- Juin 2018}{}
\item Spécialité: Service Télécom Cloud

\divider

\cvevent{Etudes préparatoires d'ingénieurie}{Institut Préparatoire des Etudes d'Ingénieurie à Tunis, IPEIT \break
\location{\textcolor{LightGrey}{Tunisie}}}{Sept 2013 -- Juin 2015}{}
\item Spécialité: Mathématiques & Physiques \\ Rang: 135/2200 dans le concours national

\divider

\cvevent{Baccalauréat Scientifique}{Lycée Pilote Kairouan \break
\location{\textcolor{LightGrey}{Tunisie}}}{Sept 2009 -- Juin 2013}{}
\item Spécialité: Mathématiques, Moyenne: 16.56/20 \\
Mention: Très Bien

\divider

\cvevent{Diplôme national de musique}{Conservatoire Régional de musique \break
\location{\textcolor{LightGrey}{Tunisie}}}{Sept 2005 -- Juin 2010}{}
\item Spécialité: Folklore tunisien, oriental \\
Mention: Bien



\end{paracol}



\end{document}
